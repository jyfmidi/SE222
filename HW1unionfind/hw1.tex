\documentclass[a4paper]{article}
\usepackage{geometry}
\usepackage{multicol}
\usepackage{setspace}
\usepackage{algorithm}
\usepackage{algorithmic}
\usepackage{amsmath}
\usepackage{listings}
\usepackage{xcolor}
%\setlength{\parskip}{0.5\baselineskip}
\geometry{left=3.0cm,right=2.5cm,top=1.0cm,bottom=2.0cm} 
\title{\textbf{Algorithmn HW1}}
\author{5140379032 JIN YI FAN}
\date{}
\begin{document}\large
\maketitle
\begin{spacing}{1.3}
$1\prec \log\log n \prec log n\prec \sqrt{n}\prec n^{3/4} \prec n \prec n\log n \prec n^2 \prec 2^n \prec n!\prec 2^{n^2}$

\section*{Problem 1: Union-find with specific canonical element}
\begin{itemize}
\item Use Weighted-Union instead of the naive Union-Find;
\item Add a new array $largest[]$ to store the biggest element included in the components containing the target element;
\item {Modify the value in $largest[]$ when unioning two trees with roots $a$ and $b$:
\begin{algorithmic}[1]
\STATE $aL \gets largest[a]$
\STATE $bL \gets largest[b]$
\IF{$aL>bL$}
\STATE $largest[b] \gets aL$
\ELSE
\STATE $largest[a] \gets bL$
\ENDIF
\end{algorithmic}
}
\item $find(i)$ function simply return the element $largest[i]$. 
\end{itemize}
\section*{Problem 2: Successor with delete}
\begin{itemize}
\item $remove(x):${
	\\We can add a new array $exist[]$ to store if the element is still in.
	\begin{algorithmic}[1]
	\STATE $exist[x] \gets false$
	\IF{$x-1$ is removed}
	\STATE $union(x,x-1)$ \ENDIF
	\IF{$x+1$ is removed}
	\STATE $union(x,x+1)$ \ENDIF
	\end{algorithmic}
}
\item $successor(x):${
	\\We can use the $find()$ function in Problem 1.
	\begin{algorithmic}[1]
	\IF{$exist[x]$ is $true$}
	\RETURN x
	\ELSE
	\RETURN $find(x)+1$ \ENDIF
	\end{algorithmic}
}
\end{itemize}

\section*{Problem 3: Union-by-height}
\textbf{Inplementation:}
\begin{lstlisting}[language={C++},basicstyle=\footnotesize,numbers=left,numberstyle=\tiny,keywordstyle=\color{blue!70},commentstyle=\color{red!50!green!50!blue!50},frame=shadowbox, rulesepcolor=\color{red!20!green!20!blue!20}] 
#include<vector>
using namespace std;
class HeightUnion{
	vector<int>id;
	vector<int>height;
public:
	HeightUnion(int N);
	int root(int p);
	void Union(int p, int q);
	bool Connected(int p, int q);
};

HeightUnion::HeightUnion(int N){
	for (int i = 0; i < N; i++){
		id.push_back(i);
		height.push_back(0);
	}
}
int HeightUnion::root(int p){
	int pPar = id[p];
	while (p != pPar){
		pPar = id[pPar];
		p = pPar;
	}
	return p;
}
void HeightUnion::Union(int p, int q){
	int rp = root(p);
	int rq = root(q);
	if (rp==rq) return;
	if (height[rp] > height[rq]) {
		id[rq] = rp;
	}
	else{
		id[rp] = rq;
		if (height[rp] == height[rq]) height[rq]++;
	}
}
bool HeightUnion::Connected(int p, int q){
	return root(p) == root(q);
}
\end{lstlisting}
\textbf{Proof:}
\\As we can see, the height of the tree unioned will change iff these two offspring trees are equally high and height will $+1$ in this case, otherwise it will remain the same.
\\So $Height(N) \leq$ (worst case)the height is updated at every $union$ operation, that is, these $N$ nodes have done $Height(N)$ symmetrical 2-to-1 merging to $1$ nodes, which is obviously $\log_2(N)$.
\end{spacing}
\end{document}